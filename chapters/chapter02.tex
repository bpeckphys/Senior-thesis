I will be examining the application of domain decomposition methods specifically applied to the Poisson problem which is a type of elliptic problem particularly important to the realm of physics. The Poisson equation $\nabla^2u = f$ involves the Laplacian and a function that only depends on its space variables \cite{Farlow}. This allows us to use a central-difference approximation for a given system. 

While there are many applications of the Poisson problem, in my research we are approximating it to model pressure as part of the over-arching goal of modeling wind through mountainous regions. Attempting to model weather without any surface interactions is a challenging problem by itself. When we begin to introduce more complex geography, the problem becomes much more complex. These large problems result in very large matrices which we would like to avoid computing using direct solvers. By decomposing these rather large problems into smaller de-coupled problems, we can achieve massive parallelization that enables us to approximate the original system in a comparably small amount of time.

A further method, which I will not discuss in detail, we utilize in my research to reduce the size of the problem is called adaptive mesh refinement. This is motivated by fluid dynamics, where the frictional force causes the highest rate of change in pressure to occur at boundaries where the fluid interacts with another object. In our situation, this means the the pressure is going to change the most where the wind interacts with the mountain. This means that we can gain a fairly accurate approximation of the pressure at points away from the surface of the mountain using significantly larger interval widths which make for much smaller problems. As we approach the surface of the mountain, however, we need much smaller interval widths to achieve the same level of accuracy. This can be done by starting with a courser mesh size as needed away from the mountain and further sub-dividing up portions of that mesh until we have a fine enough mesh to approximate the problem at the boundary of the mountainside.