The most natural place to start, although not the easiest to motivate, is the 1-dimensional problem. I will show that the equations arising from an overlapping domain decomposition using a second-order central-difference approximation along a 1-dimensional domain interface with a node centered approach ensures that the values shared at the boundary of two sub-domains are equivalent. Subsequently, I will demonstrate that this extends naturally to 2-dimensional problems.

We start by analyzing figure \ref{fig:1dgrid} which depicts a line and decomposing it into two sub-domains in the subsequent figure \ref{fig:1dgriddecomp}. Nodes $u_0$ and $u_8$ are ghost nodes used to solve along the line from $u_1$ to $u_7$. Each node can be approximated using the central-difference approximation in equation \ref{eq:3.1}.

\begin{equation}\label{eq:3.1}
	u''(x_{j}) \cong \frac{1}{h^2}[u_{j-1} - 2u_{j} + u_{j+1}]
\end{equation} 

\begin {figure}[htbp]
  \centering
    \begin{tikzpicture}
\tikzset{every node}=[draw,shape=circle]

\draw (0,0) node[circle,inner sep=2pt,label=above:$u_0$](a){} -- (1,0) 
node[circle,fill,inner sep=1pt,label=above:$u_1$](b){};
\draw (1,0) node[circle,fill,inner sep=1pt](a){} -- (2,0) 
node[circle,fill,inner sep=1pt,label=above:$u_2$](b){};
\draw (2,0) node[circle,fill,inner sep=1pt](a){} -- (3,0) 
node[circle,fill,inner sep=1pt,label=above:$u_3$](b){};
\draw (3,0) node[circle,fill,inner sep=1pt](a){} -- (4,0) 
node[circle,fill,inner sep=1pt,label=above:$u_4$](b){};
\draw (4,0) node[circle,fill,inner sep=1pt](a){} -- (5,0) 
node[circle,fill,inner sep=1pt,label=above:$u_5$](b){};
\draw (5,0) node[circle,fill,inner sep=1pt](a){} -- (6,0) 
node[circle,fill,inner sep=1pt,label=above:$u_6$](b){};
\draw (6,0) node[circle,fill,inner sep=1pt](a){} -- (7,0) 
node[circle,fill,inner sep=1pt,label=above:$u_7$](b){};
\draw (7,0) node[circle,fill,inner sep=1pt](a){} -- (8,0) 
node[circle,inner sep=2pt,label=above:$u_8$](b){};

   \end{tikzpicture}
     \caption{1-D line before domain decomposition. \label{fig:1dgrid}}
  \end{figure}

In figure \ref{fig:1dgriddecomp}, the domain has been decomposed into two parts, $\Omega_0$ and $\Omega_1$, that share the boundary $u_4^0 = u_0^1$. Each of the sub-domains then have two ghost nodes such that on each iteration $u_{-1}^1$ receives its value from $u_3^0$ and $u_5^0$ receives its value from $u_1^1$.
\begin {figure}[htbp]
  \centering
    \begin{tikzpicture}[>=triangle 60]
\tikzset{every node}=[draw,shape=circle]

\draw (0,0) node[circle,inner sep=2pt,label=above:$u_0^0$,label=left:\LARGE$\boldsymbol\Omega_0$](a){} -- (1,0) 
node[circle,fill,inner sep=1pt,label=above:$u_1^0$](b){};
\draw (1,0) node[circle,fill,inner sep=1pt](a){} -- (2,0) 
node[circle,fill,inner sep=1pt,label=above:$u_2^0$](b){};
\draw (2,0) node[circle,fill,inner sep=1pt](a){} -- (3,0) 
node[circle,fill,inner sep=1pt,label=above:$u_3^0$](b){};
\draw (3,0) node[circle,fill,inner sep=1pt](a){} -- (4,0) 
node[circle,fill,inner sep=1pt,label=above:$u_4^0$](b){};
\draw (4,0) node[circle,fill,inner sep=1pt](a){} -- (5,0) 
node[circle,inner sep=2pt,label=above:$u_5^0$](b){};

\draw[->] (3,0) -- (3,-0.9);
\draw (4,0) node[circle,fill,inner sep=1pt](a){} -- (4,-1);
\draw[->] (5,-1) -- (5,-0.1);

\draw[yshift=-1cm] (3,0) node[circle,inner sep=2pt,label=below:$u_{-1}^1$](a){} -- (4,0) 
node[circle,fill,inner sep=1pt,label=below:$u_{0}^1$](b){};
\draw[yshift=-1cm] (4,0) node[circle,fill,inner sep=1pt](a){} -- (5,0) 
node[circle,fill,inner sep=1pt,label=below:$u_{1}^1$](b){};
\draw[yshift=-1cm] (5,0) node[circle,fill,inner sep=1pt](a){} -- (6,0) 
node[circle,fill,inner sep=1pt,label=below:$u_{2}^1$](b){};
\draw[yshift=-1cm] (6,0) node[circle,fill,inner sep=1pt](a){} -- (7,0) 
node[circle,fill,inner sep=1pt,label=below:$u_{3}^1$](b){};
\draw[yshift=-1cm] (7,0) node[circle,fill,inner sep=1pt](a){} -- (8,0) 
node[circle,inner sep=2pt,label=below:$u_{4}^1$,label=right:\LARGE$\boldsymbol\Omega_1$](b){};


   \end{tikzpicture}
     \caption{1-D line after domain decomposition. \label{fig:1dgriddecomp}}
  \end{figure}

This decomposition leads to the following equations on the boundary when solving the Poisson problem where $f$ is the exact solution at each point.
\begin{alignat*}{4}
u_3^0 & {}-{} &  2u_4^0 & {}+{} & u_5^0 & {}={} & f_4^0 \\
u_{-1}^1 & {}-{} &  2u_0^1 & {}+{} & u_1^1 & {}={} & f_0^1
\end{alignat*}

From these equations we know that $f_4^0 = f_1^1$ and since at each iteration $u_{-1}^1$ receives its value from $u_3^0$ and $u_5^0$ receives its value from $u_1^1$ we see that the equations become $u_3^0 - 2u_4^0 + u_1^1 = f_4^0 = u_3^0 - 2u_0^1 + u_1^1$ and thus $u_4^0 = u_0^1$.

Now we will attempt the 2-D case. We start by making a box and decomposing it into two sub-domains as seen in the subsequent figures. I have adopted the notation of $u_{xy}$ where x is the x-coordinate and y is the y-coordinate. Nodes along $x = 0$ and $4$ as-well-as nodes along $y = 0$ and $8$ are ghost nodes. We can approximate each node using the central-difference approximation

\begin{equation}
	u''(x_{j}) \cong \frac{1}{h^2}[u_{i-1,j} + u_{i+1,j} - 4u_{i,j} + u_{i,j-1} + u_{i,j+1}]
\end{equation}

\begin {figure}[htbp]
  \centering
    \begin{tikzpicture}
\tikzset{every node}=[draw,shape=circle]

\draw (1,0) node[circle,inner sep=2pt,label=above:$u_{10}$](b){} -- (1,-4);
\draw (2,0) node[circle,inner sep=2pt,label=above:$u_{20}$](b){} -- (2,-4);
\draw (3,0) node[circle,inner sep=2pt,label=above:$u_{30}$](b){} -- (3,-4);
\draw (4,0) node[circle,inner sep=2pt,label=above:$u_{40}$](b){} -- (4,-4);
\draw (5,0) node[circle,inner sep=2pt,label=above:$u_{50}$](b){} -- (5,-4);
\draw (6,0) node[circle,inner sep=2pt,label=above:$u_{60}$](b){} -- (6,-4);
\draw (7,0) node[circle,inner sep=2pt,label=above:$u_{70}$](b){} -- (7,-4);

\draw (0,-1) node[circle,inner sep=2pt,label=above:$u_{01}$](a){} -- (1,-1) 
node[circle,fill,inner sep=1pt,label=above:$u_{11}$](b){};
\draw (1,-1) node[circle,fill,inner sep=1pt](a){} -- (2,-1) 
node[circle,fill,inner sep=1pt,label=above:$u_{21}$](b){};
\draw (2,-1) node[circle,fill,inner sep=1pt](a){} -- (3,-1) 
node[circle,fill,inner sep=1pt,label=above:$u_{31}$](b){};
\draw (3,-1) node[circle,fill,inner sep=1pt](a){} -- (4,-1) 
node[circle,fill,inner sep=1pt,label=above:$u_{41}$](b){};
\draw (4,-1) node[circle,fill,inner sep=1pt](a){} -- (5,-1) 
node[circle,fill,inner sep=1pt,label=above:$u_{51}$](b){};
\draw (5,-1) node[circle,fill,inner sep=1pt](a){} -- (6,-1) 
node[circle,fill,inner sep=1pt,label=above:$u_{61}$](b){};
\draw (6,-1) node[circle,fill,inner sep=1pt](a){} -- (7,-1) 
node[circle,fill,inner sep=1pt,label=above:$u_{71}$](b){};
\draw (7,-1) node[circle,fill,inner sep=1pt](a){} -- (8,-1) 
node[circle,inner sep=2pt,label=above:$u_{81}$](b){};

\draw (0,-2) node[circle,inner sep=2pt,label=above:$u_{02}$](a){} -- (1,-2) 
node[circle,fill,inner sep=1pt,label=above:$u_{12}$](b){};
\draw (1,-2) node[circle,fill,inner sep=1pt](a){} -- (2,-2) 
node[circle,fill,inner sep=1pt,label=above:$u_{22}$](b){};
\draw (2,-2) node[circle,fill,inner sep=1pt](a){} -- (3,-2) 
node[circle,fill,inner sep=1pt,label=above:$u_{32}$](b){};
\draw (3,-2) node[circle,fill,inner sep=1pt](a){} -- (4,-2) 
node[circle,fill,inner sep=1pt,label=above:$u_{42}$](b){};
\draw (4,-2) node[circle,fill,inner sep=1pt](a){} -- (5,-2) 
node[circle,fill,inner sep=1pt,label=above:$u_{52}$](b){};
\draw (5,-2) node[circle,fill,inner sep=1pt](a){} -- (6,-2) 
node[circle,fill,inner sep=1pt,label=above:$u_{62}$](b){};
\draw (6,-2) node[circle,fill,inner sep=1pt](a){} -- (7,-2) 
node[circle,fill,inner sep=1pt,label=above:$u_{72}$](b){};
\draw (7,-2) node[circle,fill,inner sep=1pt](a){} -- (8,-2) 
node[circle,inner sep=2pt,label=above:$u_{82}$](b){};

\draw (0,-3) node[circle,inner sep=2pt,label=above:$u_{03}$](a){} -- (1,-3) 
node[circle,fill,inner sep=1pt,label=above:$u_{13}$](b){};
\draw (1,-3) node[circle,fill,inner sep=1pt](a){} -- (2,-3) 
node[circle,fill,inner sep=1pt,label=above:$u_{23}$](b){};
\draw (2,-3) node[circle,fill,inner sep=1pt](a){} -- (3,-3) 
node[circle,fill,inner sep=1pt,label=above:$u_{33}$](b){};
\draw (3,-3) node[circle,fill,inner sep=1pt](a){} -- (4,-3) 
node[circle,fill,inner sep=1pt,label=above:$u_{43}$](b){};
\draw (4,-3) node[circle,fill,inner sep=1pt](a){} -- (5,-3) 
node[circle,fill,inner sep=1pt,label=above:$u_{53}$](b){};
\draw (5,-3) node[circle,fill,inner sep=1pt](a){} -- (6,-3) 
node[circle,fill,inner sep=1pt,label=above:$u_{63}$](b){};
\draw (6,-3) node[circle,fill,inner sep=1pt](a){} -- (7,-3) 
node[circle,fill,inner sep=1pt,label=above:$u_{73}$](b){};
\draw (7,-3) node[circle,fill,inner sep=1pt](a){} -- (8,-3) 
node[circle,inner sep=2pt,label=above:$u_{83}$](b){};

\draw (1,-4) node[circle,inner sep=2pt,label=above:$u_{14}$](b){};
\draw (2,-4) node[circle,inner sep=2pt,label=above:$u_{24}$](b){};
\draw (3,-4) node[circle,inner sep=2pt,label=above:$u_{34}$](b){};
\draw (4,-4) node[circle,inner sep=2pt,label=above:$u_{44}$](b){};
\draw (5,-4) node[circle,inner sep=2pt,label=above:$u_{54}$](b){};
\draw (6,-4) node[circle,inner sep=2pt,label=above:$u_{64}$](b){};
\draw (7,-4) node[circle,inner sep=2pt,label=above:$u_{74}$](b){};

   \end{tikzpicture}
     \caption{2-D box before domain decomposition. \label{fig:2dgrid1}}
  \end{figure}

Similar to the 1-D case, the domain has been decomposed into two parts, $\Omega_0$ and $\Omega_1$, that share the boundary $u_{4y}^0 = u_{0y}^1$. Each of the sub-domains then have ghost nodes such that for each j, $u_{-1,j}^1$ receives its value from $u_{3,j}^0$ and $u_{5,j}^0$ receives its value from $u_{1,j}^1$ every iteration.
\begin {figure}[htbp]
  \centering
    \begin{tikzpicture}
\tikzset{every node}=[draw,shape=circle]

\draw (1,0) node[circle,inner sep=2pt,label=above:$u_{10}^0$](b){} -- (1,-4);
\draw (2,0) node[circle,inner sep=2pt,label=above:$u_{20}^0$](b){} -- (2,-4);
\draw (3,0) node[circle,inner sep=2pt,label=above:$u_{30}^0$](b){} -- (3,-4);
\draw (4,0) node[circle,inner sep=2pt,label=above:$u_{40}^0$](b){} -- (4,-4);

\draw (8,0) node[circle,inner sep=2pt,label=above:$u_{00}^1$](b){} -- (8,-4);
\draw (9,0) node[circle,inner sep=2pt,label=above:$u_{10}^1$](b){} -- (9,-4);
\draw (10,0) node[circle,inner sep=2pt,label=above:$u_{20}^1$](b){} -- (10,-4);
\draw (11,0) node[circle,inner sep=2pt,label=above:$u_{30}^1$](b){} -- (11,-4);

\draw (0,-1) node[circle,inner sep=2pt,label=above:$u_{01}^0$](a){} -- (1,-1) 
node[circle,fill,inner sep=1pt,label=above:$u_{11}^0$](b){};
\draw (1,-1) node[circle,fill,inner sep=1pt](a){} -- (2,-1) 
node[circle,fill,inner sep=1pt,label=above:$u_{21}^0$](b){};
\draw (2,-1) node[circle,fill,inner sep=1pt](a){} -- (3,-1) 
node[circle,fill,inner sep=1pt,label=above:$u_{31}^0$](b){};
\draw (3,-1) node[circle,fill,inner sep=1pt](a){} -- (4,-1) 
node[circle,fill,inner sep=1pt,label=above:$u_{41}^0$](b){};
\draw (4,-1) node[circle,fill,inner sep=1pt](a){} -- (5,-1) 
node[circle,inner sep=2pt,label=above:$u_{51}^0$](b){};

\draw (7,-1) node[circle,inner sep=2pt,label=above:$u_{-11}^1$](a){} -- (8,-1) 
node[circle,fill,inner sep=1pt,label=above:$u_{01}^1$](b){};
\draw (8,-1) node[circle,fill,inner sep=1pt](a){} -- (9,-1) 
node[circle,fill,inner sep=1pt,label=above:$u_{11}^1$](b){};
\draw (9,-1) node[circle,fill,inner sep=1pt](a){} -- (10,-1) 
node[circle,fill,inner sep=1pt,label=above:$u_{21}^1$](b){};
\draw (10,-1) node[circle,fill,inner sep=1pt](a){} -- (11,-1) 
node[circle,fill,inner sep=1pt,label=above:$u_{31}^1$](b){};
\draw (11,-1) node[circle,fill,inner sep=1pt](a){} -- (12,-1) 
node[circle,inner sep=2pt,label=above:$u_{41}^1$](b){};

\draw (0,-2) node[circle,inner sep=2pt,label=above:$u_{02}^0$,label=left:\LARGE$\boldsymbol\Omega_0$](a){} -- (1,-2) 
node[circle,fill,inner sep=1pt,label=above:$u_{12}^0$](b){};
\draw (1,-2) node[circle,fill,inner sep=1pt](a){} -- (2,-2) 
node[circle,fill,inner sep=1pt,label=above:$u_{22}^0$](b){};
\draw (2,-2) node[circle,fill,inner sep=1pt](a){} -- (3,-2) 
node[circle,fill,inner sep=1pt,label=above:$u_{32}^0$](b){};
\draw (3,-2) node[circle,fill,inner sep=1pt](a){} -- (4,-2) 
node[circle,fill,inner sep=1pt,label=above:$u_{42}^0$](b){};
\draw (4,-2) node[circle,fill,inner sep=1pt](a){} -- (5,-2) 
node[circle,inner sep=2pt,label=above:$u_{52}^0$](b){};

\draw (7,-2) node[circle,inner sep=2pt,label=above:$u_{-12}^1$](a){} -- (8,-2) 
node[circle,fill,inner sep=1pt,label=above:$u_{02}^1$](b){};
\draw (8,-2) node[circle,fill,inner sep=1pt](a){} -- (9,-2) 
node[circle,fill,inner sep=1pt,label=above:$u_{12}^1$](b){};
\draw (9,-2) node[circle,fill,inner sep=1pt](a){} -- (10,-2) 
node[circle,fill,inner sep=1pt,label=above:$u_{22}^1$](b){};
\draw (10,-2) node[circle,fill,inner sep=1pt](a){} -- (11,-2) 
node[circle,fill,inner sep=1pt,label=above:$u_{32}^1$](b){};
\draw (11,-2) node[circle,fill,inner sep=1pt](a){} -- (12,-2) 
node[circle,inner sep=2pt,label=above:$u_{42}^1$,label=right:\LARGE$\boldsymbol\Omega_1$](b){};

\draw (0,-3) node[circle,inner sep=2pt,label=above:$u_{03}^0$](a){} -- (1,-3) 
node[circle,fill,inner sep=1pt,label=above:$u_{13}^0$](b){};
\draw (1,-3) node[circle,fill,inner sep=1pt](a){} -- (2,-3) 
node[circle,fill,inner sep=1pt,label=above:$u_{23}^0$](b){};
\draw (2,-3) node[circle,fill,inner sep=1pt](a){} -- (3,-3) 
node[circle,fill,inner sep=1pt,label=above:$u_{33}^0$](b){};
\draw (3,-3) node[circle,fill,inner sep=1pt](a){} -- (4,-3) 
node[circle,fill,inner sep=1pt,label=above:$u_{43}^0$](b){};
\draw (4,-3) node[circle,fill,inner sep=1pt](a){} -- (5,-3) 
node[circle,inner sep=2pt,label=above:$u_{53}^0$](b){};

\draw (7,-3) node[circle,inner sep=2pt,label=above:$u_{-13}^1$](a){} -- (8,-3) 
node[circle,fill,inner sep=1pt,label=above:$u_{03}^1$](b){};
\draw (8,-3) node[circle,fill,inner sep=1pt](a){} -- (9,-3) 
node[circle,fill,inner sep=1pt,label=above:$u_{13}^1$](b){};
\draw (9,-3) node[circle,fill,inner sep=1pt](a){} -- (10,-3) 
node[circle,fill,inner sep=1pt,label=above:$u_{23}^1$](b){};
\draw (10,-3) node[circle,fill,inner sep=1pt](a){} -- (11,-3) 
node[circle,fill,inner sep=1pt,label=above:$u_{33}^1$](b){};
\draw (11,-3) node[circle,fill,inner sep=1pt](a){} -- (12,-3) 
node[circle,inner sep=2pt,label=above:$u_{43}^1$](b){};

\draw (1,-4) node[circle,inner sep=2pt,label=above:$u_{14}^0$](b){};
\draw (2,-4) node[circle,inner sep=2pt,label=above:$u_{24}^0$](b){};
\draw (3,-4) node[circle,inner sep=2pt,label=above:$u_{34}^0$](b){};
\draw (4,-4) node[circle,inner sep=2pt,label=above:$u_{44}^0$](b){};

\draw (8,-4) node[circle,inner sep=2pt,label=above:$u_{04}^1$](b){};
\draw (9,-4) node[circle,inner sep=2pt,label=above:$u_{14}^1$](b){};
\draw (10,-4) node[circle,inner sep=2pt,label=above:$u_{24}^1$](b){};
\draw (11,-4) node[circle,inner sep=2pt,label=above:$u_{34}^1$](b){};

   \end{tikzpicture}
     \caption{2-D box after domain decomposition. \label{fig:2dgriddecomp1}}
  \end{figure}

\pagebreak
The resulting decomposition yields the following equations on the boundary for the Poisson problem.
\begin{alignat*}{6}
u_{31}^0 & {}+{} & u_{51}^0 & {}+{} & u_{40}^0 & {}+{} &  u_{42}^0 & {}-{} &  4u_{41}^0 & {}={} & f_{41}^0 \\
u_{-11}^1 & {}+{} & u_{11}^1 & {}+{} & u_{00}^1 & {}+{} &  u_{02}^1 & {}-{} &  4u_{01}^1 & {}={} & f_{01}^1\\
u_{32}^0 & {}+{} & u_{52}^0 & {}+{} & u_{41}^0 & {}+{} &  u_{43}^0 & {}-{} &  4u_{42}^0 & {}={} & f_{42}^0 \\
u_{-12}^1 & {}+{} & u_{12}^1 & {}+{} & u_{01}^1 & {}+{} &  u_{03}^1 & {}-{} &  4u_{02}^1 & {}={} & f_{02}^1\\
u_{33}^0 & {}+{} & u_{53}^0 & {}+{} & u_{42}^0 & {}+{} &  u_{44}^0 & {}-{} &  4u_{43}^0 & {}={} & f_{43}^0 \\
u_{-13}^1 & {}+{} & u_{13}^1 & {}+{} & u_{02}^1 & {}+{} &  u_{04}^1 & {}-{} &  4u_{03}^1 & {}={} & f_{03}^1
\end{alignat*}

From these equations we know that for each j, $f_{4,j}^0 = f_{0,j}^1$ and since at each iteration $u_{-1,j}^1$ receives its value from $u_{3,j}^0$ and $u_{5,j}^0$ receives its value from $u_{1,j}^1$ we see that after reducing, the equations become:
\begin{alignat*}{11}
u_{40}^0 & {}+{} &  u_{42}^0 & {}-{} &  4u_{41}^0 & {}={} & u_{00}^1 & {}+{} &  u_{02}^1 & {}-{} &  4u_{01}^1\\
u_{41}^0 & {}+{} &  u_{43}^0 & {}-{} &  4u_{42}^0 & {}={} & u_{01}^1 & {}+{} &  u_{03}^1 & {}-{} &  4u_{02}^1\\
u_{42}^0 & {}+{} &  u_{44}^0 & {}-{} &  4u_{43}^0 & {}={} & u_{02}^1 & {}+{} &  u_{04}^1 & {}-{} &  4u_{03}^1
\end{alignat*}

Furthermore, since we have boundary conditions, the ghost nodes on the boundary are known to be equivalent. That is, $u_{40}^0 = u_{00}^1$ and $u_{44}^0 = u_{04}^1$ which gives the following system of equations.
\begin{alignat}{6}
\label{eq:3.3} && u_{42}^0 & {}-{} &  4u_{41}^0 & {}={} & u_{02}^1 & {}-{} &  4u_{01}^1 && \\
\label{eq:3.4}u_{41}^0 & {}+{} &  u_{43}^0 & {}-{} &  4u_{42}^0 & {}={} & u_{01}^1 & {}+{} &  u_{03}^1 & {}-{} &  4u_{02}^1\\
\label{eq:3.5}&& u_{42}^0 & {}+{} &  4u_{43}^0 & {}={} & u_{02}^1 & {}+{} &  4u_{03}^1 &&
\end{alignat}

Now if we both solve \ref{eq:3.3} and \ref{eq:3.5} for $u_{02}^1$ and $u_{42}^0$ and plug into \ref{eq:3.4}, as-well-as subtract \ref{eq:3.5} from \ref{eq:3.3} we get
\begin{alignat*}{4} 
	u_{41}^0 & {}+{} &  u_{43}^0 & {}={} & u_{01}^1 & {}+{} &  u_{03}^1\\
	-u_{41}^0 & {}+{} &  u_{43}^0 & {}={} & -u_{01}^1 & {}+{} &  u_{03}^1
\end{alignat*}

Adding and subtracting these equations together and then plugging into \ref{eq:3.4} we arrive at the final result we are looking for. 
\begin{alignat*}{2} 
	u_{41}^0 & {}={} & u_{01}^1\\
	u_{42}^0 & {}={} & u_{02}^1\\
	u_{43}^0 & {}={} & u_{03}^1
\end{alignat*}

For the sake of thoroughness, we will extend this to 4 sub-domains in 4 quadrants as shown in the next two figures. Similar to our proof that the boundary values between $\Omega_0$ and $\Omega_1$ are equivalent, it can be shown that the boundary values between $\Omega_0$ and $\Omega_2$ are equivalent as are those between $\Omega_1$ and $\Omega_3$. What remains then is to show that the boundary value shared by all 4 nodes is equivalent for each; that is, $u_{43}^0 = u_{03}^1 = u_{40}^2 = u_{00}^3$.
\begin {figure}[htbp]
  \centering
    \begin{tikzpicture}
\tikzset{every node}=[draw,shape=circle]

\draw (1,0) node[circle,inner sep=2pt,label=above:$u_{10}$](b){} -- (1,-6)
	node[circle,inner sep=2pt,label=above:$u_{16}$](b){};
\draw (2,0) node[circle,inner sep=2pt,label=above:$u_{20}$](b){} -- (2,-6)
	node[circle,inner sep=2pt,label=above:$u_{26}$](b){};
\draw (3,0) node[circle,inner sep=2pt,label=above:$u_{30}$](b){} -- (3,-6)
	node[circle,inner sep=2pt,label=above:$u_{36}$](b){};
\draw (4,0) node[circle,inner sep=2pt,label=above:$u_{40}$](b){} -- (4,-6)
	node[circle,inner sep=2pt,label=above:$u_{46}$](b){};
\draw (5,0) node[circle,inner sep=2pt,label=above:$u_{50}$](b){} -- (5,-6)
	node[circle,inner sep=2pt,label=above:$u_{56}$](b){};
\draw (6,0) node[circle,inner sep=2pt,label=above:$u_{60}$](b){} -- (6,-6)
	node[circle,inner sep=2pt,label=above:$u_{66}$](b){};
\draw (7,0) node[circle,inner sep=2pt,label=above:$u_{70}$](b){} -- (7,-6)
	node[circle,inner sep=2pt,label=above:$u_{76}$](b){};

\draw (0,-1) node[circle,inner sep=2pt,label=above:$u_{01}$](a){} -- (1,-1) 
node[circle,fill,inner sep=1pt,label=above:$u_{11}$](b){};
\draw (1,-1) node[circle,fill,inner sep=1pt](a){} -- (2,-1) 
node[circle,fill,inner sep=1pt,label=above:$u_{21}$](b){};
\draw (2,-1) node[circle,fill,inner sep=1pt](a){} -- (3,-1) 
node[circle,fill,inner sep=1pt,label=above:$u_{31}$](b){};
\draw (3,-1) node[circle,fill,inner sep=1pt](a){} -- (4,-1) 
node[circle,fill,inner sep=1pt,label=above:$u_{41}$](b){};
\draw (4,-1) node[circle,fill,inner sep=1pt](a){} -- (5,-1) 
node[circle,fill,inner sep=1pt,label=above:$u_{51}$](b){};
\draw (5,-1) node[circle,fill,inner sep=1pt](a){} -- (6,-1) 
node[circle,fill,inner sep=1pt,label=above:$u_{61}$](b){};
\draw (6,-1) node[circle,fill,inner sep=1pt](a){} -- (7,-1) 
node[circle,fill,inner sep=1pt,label=above:$u_{71}$](b){};
\draw (7,-1) node[circle,fill,inner sep=1pt](a){} -- (8,-1) 
node[circle,inner sep=2pt,label=above:$u_{81}$](b){};

\draw (0,-2) node[circle,inner sep=2pt,label=above:$u_{02}$](a){} -- (1,-2) 
node[circle,fill,inner sep=1pt,label=above:$u_{12}$](b){};
\draw (1,-2) node[circle,fill,inner sep=1pt](a){} -- (2,-2) 
node[circle,fill,inner sep=1pt,label=above:$u_{22}$](b){};
\draw (2,-2) node[circle,fill,inner sep=1pt](a){} -- (3,-2) 
node[circle,fill,inner sep=1pt,label=above:$u_{32}$](b){};
\draw (3,-2) node[circle,fill,inner sep=1pt](a){} -- (4,-2) 
node[circle,fill,inner sep=1pt,label=above:$u_{42}$](b){};
\draw (4,-2) node[circle,fill,inner sep=1pt](a){} -- (5,-2) 
node[circle,fill,inner sep=1pt,label=above:$u_{52}$](b){};
\draw (5,-2) node[circle,fill,inner sep=1pt](a){} -- (6,-2) 
node[circle,fill,inner sep=1pt,label=above:$u_{62}$](b){};
\draw (6,-2) node[circle,fill,inner sep=1pt](a){} -- (7,-2) 
node[circle,fill,inner sep=1pt,label=above:$u_{72}$](b){};
\draw (7,-2) node[circle,fill,inner sep=1pt](a){} -- (8,-2) 
node[circle,inner sep=2pt,label=above:$u_{82}$](b){};

\draw (0,-3) node[circle,inner sep=2pt,label=above:$u_{03}$](a){} -- (1,-3) 
node[circle,fill,inner sep=1pt,label=above:$u_{13}$](b){};
\draw (1,-3) node[circle,fill,inner sep=1pt](a){} -- (2,-3) 
node[circle,fill,inner sep=1pt,label=above:$u_{23}$](b){};
\draw (2,-3) node[circle,fill,inner sep=1pt](a){} -- (3,-3) 
node[circle,fill,inner sep=1pt,label=above:$u_{33}$](b){};
\draw (3,-3) node[circle,fill,inner sep=1pt](a){} -- (4,-3) 
node[circle,fill,inner sep=1pt,label=above:$u_{43}$](b){};
\draw (4,-3) node[circle,fill,inner sep=1pt](a){} -- (5,-3) 
node[circle,fill,inner sep=1pt,label=above:$u_{53}$](b){};
\draw (5,-3) node[circle,fill,inner sep=1pt](a){} -- (6,-3) 
node[circle,fill,inner sep=1pt,label=above:$u_{63}$](b){};
\draw (6,-3) node[circle,fill,inner sep=1pt](a){} -- (7,-3) 
node[circle,fill,inner sep=1pt,label=above:$u_{73}$](b){};
\draw (7,-3) node[circle,fill,inner sep=1pt](a){} -- (8,-3) 
node[circle,inner sep=2pt,label=above:$u_{83}$](b){};

\draw (0,-4) node[circle,inner sep=2pt,label=above:$u_{04}$](a){} -- (1,-4) 
node[circle,fill,inner sep=1pt,label=above:$u_{14}$](b){};
\draw (1,-4) node[circle,fill,inner sep=1pt](a){} -- (2,-4) 
node[circle,fill,inner sep=1pt,label=above:$u_{24}$](b){};
\draw (2,-4) node[circle,fill,inner sep=1pt](a){} -- (3,-4) 
node[circle,fill,inner sep=1pt,label=above:$u_{34}$](b){};
\draw (3,-4) node[circle,fill,inner sep=1pt](a){} -- (4,-4) 
node[circle,fill,inner sep=1pt,label=above:$u_{44}$](b){};
\draw (4,-4) node[circle,fill,inner sep=1pt](a){} -- (5,-4) 
node[circle,fill,inner sep=1pt,label=above:$u_{54}$](b){};
\draw (5,-4) node[circle,fill,inner sep=1pt](a){} -- (6,-4) 
node[circle,fill,inner sep=1pt,label=above:$u_{64}$](b){};
\draw (6,-4) node[circle,fill,inner sep=1pt](a){} -- (7,-4) 
node[circle,fill,inner sep=1pt,label=above:$u_{74}$](b){};
\draw (7,-4) node[circle,fill,inner sep=1pt](a){} -- (8,-4) 
node[circle,inner sep=2pt,label=above:$u_{84}$](b){};

\draw (0,-5) node[circle,inner sep=2pt,label=above:$u_{05}$](a){} -- (1,-5) 
node[circle,fill,inner sep=1pt,label=above:$u_{15}$](b){};
\draw (1,-5) node[circle,fill,inner sep=1pt](a){} -- (2,-5) 
node[circle,fill,inner sep=1pt,label=above:$u_{25}$](b){};
\draw (2,-5) node[circle,fill,inner sep=1pt](a){} -- (3,-5) 
node[circle,fill,inner sep=1pt,label=above:$u_{35}$](b){};
\draw (3,-5) node[circle,fill,inner sep=1pt](a){} -- (4,-5) 
node[circle,fill,inner sep=1pt,label=above:$u_{45}$](b){};
\draw (4,-5) node[circle,fill,inner sep=1pt](a){} -- (5,-5) 
node[circle,fill,inner sep=1pt,label=above:$u_{55}$](b){};
\draw (5,-5) node[circle,fill,inner sep=1pt](a){} -- (6,-5) 
node[circle,fill,inner sep=1pt,label=above:$u_{65}$](b){};
\draw (6,-5) node[circle,fill,inner sep=1pt](a){} -- (7,-5) 
node[circle,fill,inner sep=1pt,label=above:$u_{75}$](b){};
\draw (7,-5) node[circle,fill,inner sep=1pt](a){} -- (8,-5) 
node[circle,inner sep=2pt,label=above:$u_{85}$](b){};


   \end{tikzpicture}
     \caption{2-D box before domain decomposition. \label{fig:2dgrid2}}
  \end{figure}
\begin {figure}[htbp]
  \centering
    \begin{tikzpicture}
\tikzset{every node}=[draw,shape=circle]

%% Verticle Lines
%% Omega 0
\draw (1,0) node[circle,inner sep=2pt,label=above:$u_{10}^0$](b){} -- (1,-4)
	node[circle,inner sep=2pt,label=above:$u_{14}^0$](b){};
\draw (2,0) node[circle,inner sep=2pt,label=above:$u_{20}^0$](b){} -- (2,-4)
	node[circle,inner sep=2pt,label=above:$u_{24}^0$](b){};
\draw (3,0) node[circle,inner sep=2pt,label=above:$u_{30}^0$](b){} -- (3,-4)
	node[circle,inner sep=2pt,label=above:$u_{34}^0$](b){};
\draw (4,0) node[circle,inner sep=2pt,label=above:$u_{40}^0$](b){} -- (4,-4)
	node[circle,inner sep=2pt,label=above:$u_{44}^0$](b){};
%% Omega 1
\draw (8,0) node[circle,inner sep=2pt,label=above:$u_{00}^1$](b){} -- (8,-4)
	node[circle,inner sep=2pt,label=above:$u_{04}^1$](b){};
\draw (9,0) node[circle,inner sep=2pt,label=above:$u_{10}^1$](b){} -- (9,-4)
	node[circle,inner sep=2pt,label=above:$u_{14}^1$](b){};
\draw (10,0) node[circle,inner sep=2pt,label=above:$u_{20}^1$](b){} -- (10,-4)
	node[circle,inner sep=2pt,label=above:$u_{24}^1$](b){};
\draw (11,0) node[circle,inner sep=2pt,label=above:$u_{30}^1$](b){} -- (11,-4)
	node[circle,inner sep=2pt,label=above:$u_{34}^1$](b){};
%% Omega 2
\draw (1,-6) node[circle,inner sep=2pt,label=above:$u_{1-1}^2$](b){} -- (1,-10)
	node[circle,inner sep=2pt,label=above:$u_{13}^2$](b){};
\draw (2,-6) node[circle,inner sep=2pt,label=above:$u_{2-1}^2$](b){} -- (2,-10)
	node[circle,inner sep=2pt,label=above:$u_{23}^2$](b){};
\draw (3,-6) node[circle,inner sep=2pt,label=above:$u_{3-1}^2$](b){} -- (3,-10)
	node[circle,inner sep=2pt,label=above:$u_{33}^2$](b){};
\draw (4,-6) node[circle,inner sep=2pt,label=above:$u_{4-1}^2$](b){} -- (4,-10)
	node[circle,inner sep=2pt,label=above:$u_{43}^2$](b){};
%% Omega 3
\draw (8,-6) node[circle,inner sep=2pt,label=above:$u_{0-1}^3$](b){} -- (8,-10)
	node[circle,inner sep=2pt,label=above:$u_{03}^3$](b){};
\draw (9,-6) node[circle,inner sep=2pt,label=above:$u_{1-1}^3$](b){} -- (9,-10)
	node[circle,inner sep=2pt,label=above:$u_{13}^3$](b){};
\draw (10,-6) node[circle,inner sep=2pt,label=above:$u_{2-1}^3$](b){} -- (10,-10)
	node[circle,inner sep=2pt,label=above:$u_{23}^3$](b){};
\draw (11,-6) node[circle,inner sep=2pt,label=above:$u_{3-1}^3$](b){} -- (11,-10)
	node[circle,inner sep=2pt,label=above:$u_{33}^3$](b){};
	
%% Horizontal Lines
%% Omega 0
\draw (0,-1) node[circle,inner sep=2pt,label=above:$u_{01}^0$](a){} -- (1,-1) 
node[circle,fill,inner sep=1pt,label=above:$u_{11}^0$](b){};
\draw (1,-1) node[circle,fill,inner sep=1pt](a){} -- (2,-1) 
node[circle,fill,inner sep=1pt,label=above:$u_{21}^0$](b){};
\draw (2,-1) node[circle,fill,inner sep=1pt](a){} -- (3,-1) 
node[circle,fill,inner sep=1pt,label=above:$u_{31}^0$](b){};
\draw (3,-1) node[circle,fill,inner sep=1pt](a){} -- (4,-1) 
node[circle,fill,inner sep=1pt,label=above:$u_{41}^0$](b){};
\draw (4,-1) node[circle,fill,inner sep=1pt](a){} -- (5,-1) 
node[circle,inner sep=2pt,label=above:$u_{51}^0$](b){};

\draw (0,-2) node[circle,inner sep=2pt,label=above:$u_{02}^0$,label=left:\LARGE$\boldsymbol\Omega_0$](a){} -- (1,-2) 
node[circle,fill,inner sep=1pt,label=above:$u_{12}^0$](b){};
\draw (1,-2) node[circle,fill,inner sep=1pt](a){} -- (2,-2) 
node[circle,fill,inner sep=1pt,label=above:$u_{22}^0$](b){};
\draw (2,-2) node[circle,fill,inner sep=1pt](a){} -- (3,-2) 
node[circle,fill,inner sep=1pt,label=above:$u_{32}^0$](b){};
\draw (3,-2) node[circle,fill,inner sep=1pt](a){} -- (4,-2) 
node[circle,fill,inner sep=1pt,label=above:$u_{42}^0$](b){};
\draw (4,-2) node[circle,fill,inner sep=1pt](a){} -- (5,-2) 
node[circle,inner sep=2pt,label=above:$u_{52}^0$](b){};

\draw (0,-3) node[circle,inner sep=2pt,label=above:$u_{03}^0$](a){} -- (1,-3) 
node[circle,fill,inner sep=1pt,label=above:$u_{13}^0$](b){};
\draw (1,-3) node[circle,fill,inner sep=1pt](a){} -- (2,-3) 
node[circle,fill,inner sep=1pt,label=above:$u_{23}^0$](b){};
\draw (2,-3) node[circle,fill,inner sep=1pt](a){} -- (3,-3) 
node[circle,fill,inner sep=1pt,label=above:$u_{33}^0$](b){};
\draw (3,-3) node[circle,fill,inner sep=1pt](a){} -- (4,-3) 
node[circle,fill,inner sep=1pt,label=above:$u_{43}^0$](b){};
\draw (4,-3) node[circle,fill,inner sep=1pt](a){} -- (5,-3) 
node[circle,inner sep=2pt,label=above:$u_{53}^0$](b){};

%% Omega 1
\draw (7,-1) node[circle,inner sep=2pt,label=above:$u_{-11}^1$](a){} -- (8,-1) 
node[circle,fill,inner sep=1pt,label=above:$u_{01}^1$](b){};
\draw (8,-1) node[circle,fill,inner sep=1pt](a){} -- (9,-1) 
node[circle,fill,inner sep=1pt,label=above:$u_{11}^1$](b){};
\draw (9,-1) node[circle,fill,inner sep=1pt](a){} -- (10,-1) 
node[circle,fill,inner sep=1pt,label=above:$u_{21}^1$](b){};
\draw (10,-1) node[circle,fill,inner sep=1pt](a){} -- (11,-1) 
node[circle,fill,inner sep=1pt,label=above:$u_{31}^1$](b){};
\draw (11,-1) node[circle,fill,inner sep=1pt](a){} -- (12,-1) 
node[circle,inner sep=2pt,label=above:$u_{41}^1$](b){};

\draw (7,-2) node[circle,inner sep=2pt,label=above:$u_{-12}^1$](a){} -- (8,-2) 
node[circle,fill,inner sep=1pt,label=above:$u_{02}^1$](b){};
\draw (8,-2) node[circle,fill,inner sep=1pt](a){} -- (9,-2) 
node[circle,fill,inner sep=1pt,label=above:$u_{12}^1$](b){};
\draw (9,-2) node[circle,fill,inner sep=1pt](a){} -- (10,-2) 
node[circle,fill,inner sep=1pt,label=above:$u_{22}^1$](b){};
\draw (10,-2) node[circle,fill,inner sep=1pt](a){} -- (11,-2) 
node[circle,fill,inner sep=1pt,label=above:$u_{32}^1$](b){};
\draw (11,-2) node[circle,fill,inner sep=1pt](a){} -- (12,-2) 
node[circle,inner sep=2pt,label=above:$u_{42}^1$,label=right:\LARGE$\boldsymbol\Omega_1$](b){};

\draw (7,-3) node[circle,inner sep=2pt,label=above:$u_{-13}^1$](a){} -- (8,-3) 
node[circle,fill,inner sep=1pt,label=above:$u_{03}^1$](b){};
\draw (8,-3) node[circle,fill,inner sep=1pt](a){} -- (9,-3) 
node[circle,fill,inner sep=1pt,label=above:$u_{13}^1$](b){};
\draw (9,-3) node[circle,fill,inner sep=1pt](a){} -- (10,-3) 
node[circle,fill,inner sep=1pt,label=above:$u_{23}^1$](b){};
\draw (10,-3) node[circle,fill,inner sep=1pt](a){} -- (11,-3) 
node[circle,fill,inner sep=1pt,label=above:$u_{33}^1$](b){};
\draw (11,-3) node[circle,fill,inner sep=1pt](a){} -- (12,-3) 
node[circle,inner sep=2pt,label=above:$u_{43}^1$](b){};

%% Omega 2
\draw (0,-7) node[circle,inner sep=2pt,label=above:$u_{00}^2$](a){} -- (1,-7) 
node[circle,fill,inner sep=1pt,label=above:$u_{10}^2$](b){};
\draw (1,-7) node[circle,fill,inner sep=1pt](a){} -- (2,-7) 
node[circle,fill,inner sep=1pt,label=above:$u_{20}^2$](b){};
\draw (2,-7) node[circle,fill,inner sep=1pt](a){} -- (3,-7) 
node[circle,fill,inner sep=1pt,label=above:$u_{30}^2$](b){};
\draw (3,-7) node[circle,fill,inner sep=1pt](a){} -- (4,-7) 
node[circle,fill,inner sep=1pt,label=above:$u_{40}^2$](b){};
\draw (4,-7) node[circle,fill,inner sep=1pt](a){} -- (5,-7) 
node[circle,inner sep=2pt,label=above:$u_{50}^2$](b){};

\draw (0,-8) node[circle,inner sep=2pt,label=above:$u_{01}^2$,label=left:\LARGE$\boldsymbol\Omega_2$](a){} -- (1,-8) 
node[circle,fill,inner sep=1pt,label=above:$u_{11}^2$](b){};
\draw (1,-8) node[circle,fill,inner sep=1pt](a){} -- (2,-8) 
node[circle,fill,inner sep=1pt,label=above:$u_{21}^2$](b){};
\draw (2,-8) node[circle,fill,inner sep=1pt](a){} -- (3,-8) 
node[circle,fill,inner sep=1pt,label=above:$u_{31}^2$](b){};
\draw (3,-8) node[circle,fill,inner sep=1pt](a){} -- (4,-8) 
node[circle,fill,inner sep=1pt,label=above:$u_{41}^2$](b){};
\draw (4,-8) node[circle,fill,inner sep=1pt](a){} -- (5,-8) 
node[circle,inner sep=2pt,label=above:$u_{51}^2$](b){};

\draw (0,-9) node[circle,inner sep=2pt,label=above:$u_{02}^2$](a){} -- (1,-9) 
node[circle,fill,inner sep=1pt,label=above:$u_{12}^2$](b){};
\draw (1,-9) node[circle,fill,inner sep=1pt](a){} -- (2,-9) 
node[circle,fill,inner sep=1pt,label=above:$u_{22}^2$](b){};
\draw (2,-9) node[circle,fill,inner sep=1pt](a){} -- (3,-9) 
node[circle,fill,inner sep=1pt,label=above:$u_{32}^2$](b){};
\draw (3,-9) node[circle,fill,inner sep=1pt](a){} -- (4,-9) 
node[circle,fill,inner sep=1pt,label=above:$u_{42}^2$](b){};
\draw (4,-9) node[circle,fill,inner sep=1pt](a){} -- (5,-9) 
node[circle,inner sep=2pt,label=above:$u_{52}^2$](b){};

%% Omega 3
\draw (7,-7) node[circle,inner sep=2pt,label=above:$u_{-10}^3$](a){} -- (8,-7) 
node[circle,fill,inner sep=1pt,label=above:$u_{00}^3$](b){};
\draw (8,-7) node[circle,fill,inner sep=1pt](a){} -- (9,-7) 
node[circle,fill,inner sep=1pt,label=above:$u_{10}^3$](b){};
\draw (9,-7) node[circle,fill,inner sep=1pt](a){} -- (10,-7) 
node[circle,fill,inner sep=1pt,label=above:$u_{20}^3$](b){};
\draw (10,-7) node[circle,fill,inner sep=1pt](a){} -- (11,-7) 
node[circle,fill,inner sep=1pt,label=above:$u_{30}^3$](b){};
\draw (11,-7) node[circle,fill,inner sep=1pt](a){} -- (12,-7) 
node[circle,inner sep=2pt,label=above:$u_{40}^3$](b){};

\draw (7,-8) node[circle,inner sep=2pt,label=above:$u_{-11}^3$](a){} -- (8,-8) 
node[circle,fill,inner sep=1pt,label=above:$u_{01}^3$](b){};
\draw (8,-8) node[circle,fill,inner sep=1pt](a){} -- (9,-8) 
node[circle,fill,inner sep=1pt,label=above:$u_{11}^3$](b){};
\draw (9,-8) node[circle,fill,inner sep=1pt](a){} -- (10,-8) 
node[circle,fill,inner sep=1pt,label=above:$u_{21}^3$](b){};
\draw (10,-8) node[circle,fill,inner sep=1pt](a){} -- (11,-8) 
node[circle,fill,inner sep=1pt,label=above:$u_{31}^3$](b){};
\draw (11,-8) node[circle,fill,inner sep=1pt](a){} -- (12,-8) 
node[circle,inner sep=2pt,label=above:$u_{41}^3$,label=right:\LARGE$\boldsymbol\Omega_3$](b){};

\draw (7,-9) node[circle,inner sep=2pt,label=above:$u_{-12}^3$](a){} -- (8,-9) 
node[circle,fill,inner sep=1pt,label=above:$u_{02}^3$](b){};
\draw (8,-9) node[circle,fill,inner sep=1pt](a){} -- (9,-9) 
node[circle,fill,inner sep=1pt,label=above:$u_{12}^3$](b){};
\draw (9,-9) node[circle,fill,inner sep=1pt](a){} -- (10,-9) 
node[circle,fill,inner sep=1pt,label=above:$u_{22}^3$](b){};
\draw (10,-9) node[circle,fill,inner sep=1pt](a){} -- (11,-9) 
node[circle,fill,inner sep=1pt,label=above:$u_{32}^3$](b){};
\draw (11,-9) node[circle,fill,inner sep=1pt](a){} -- (12,-9) 
node[circle,inner sep=2pt,label=above:$u_{42}^3$](b){};


   \end{tikzpicture}
     \caption{2-D box before domain decomposition. \label{fig:2dgriddecomp2}}
  \end{figure}

\pagebreak
Taking the values of the ghost nodes from their adjacent sub-domains and knowing that the function is equivalent on $f_{43}^0, f_{03}^1, f_{40}^2,$ and $f_{00}^3$, we obtain the following system of equations.
\begin{alignat*}{6} 
	& u_{42}^0 & {}+{} &  u_{41}^2 & {}+{} &  u_{33}^0 & {}+{} &  u_{13}^1 & {}-{} &  4u_{43}^0\\
	{}={} & u_{02}^1 & {}+{} &  u_{01}^3 & {}+{} &  u_{33}^0 & {}+{} &  u_{13}^1 & {}-{} &  4u_{03}^1\\
	{}={} & u_{42}^0 & {}+{} &  u_{41}^2 & {}+{} &  u_{30}^2 & {}+{} &  u_{10}^3 & {}-{} &  4u_{40}^2\\
	{}={} & u_{02}^1 & {}+{} &  u_{01}^3 & {}+{} &  u_{30}^2 & {}+{} &  u_{10}^3 & {}-{} &  4u_{00}^3
\end{alignat*}

Similarly, we know that ghost nodes sharing a boundary have the same value as shown below.
\begin{alignat*}{4} 
	u_{53}^0 & {}={} & u_{50}^2 & {}={} & u_{13}^1 & {}={} & u_{10}^3\\
	u_{44}^0 & {}={} & u_{04}^1 & {}={} & u_{41}^2 & {}={} & u_{01}^3\\
	u_{4-1}^2 & {}={} & u_{0-1}^3 & {}={} & u_{42}^0 & {}={} & u_{02}^1\\
	u_{-13}^1 & {}={} & u_{-10}^3 & {}={} & u_{33}^0 & {}={} & u_{30}^2
\end{alignat*}

Plugging these in, we get that $u_{43}^0 = u_{03}^1 = u_{40}^2 = u_{00}^3$ which is what we expected.





