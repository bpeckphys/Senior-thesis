Using an overlapping domain method works well for some situations; however, there are situations that arise in which we would like to de-couple the sub-domains entirely. To motivate this method I will switch to a cell-centered approach. Starting in 1-D again we can visualize this using figure \ref{fig:4.1} which depicts a straight line with nodes at the center of each cell. We use the same central difference approximation we showed in equation \ref{eq:3.1}. The system boundaries are indicated at $u|_{x=0}$ and $u|_{x=1}$. For the non-overlapping case we define the boundary between two sub-domains as $\Gamma$. The value of $u$ on the boundary is then $\gamma$, so I have indicated the interior boundary $\Gamma$ at ${u|_{\Gamma} = \gamma}$.

\begin {figure}[htbp]
  \centering
    \begin{tikzpicture}
\tikzset{every node}=[draw,shape=circle]

\draw (0.5,0.5)  -- (0.5,-0.5);
\draw (0.5,0.2) node[draw=none,rectangle,label=above:$0$](a){};
\draw (0.5,0) node[draw=none,rectangle,label=below:$u|_{x=0}$](a){};
\draw (0,0) node[circle,inner sep=2pt,label=above:$u_0$](a){} -- (1,0) 
node[circle,fill,inner sep=1pt,label=above:$u_1$](b){};
\draw (1,0) node[circle,fill,inner sep=1pt](a){} -- (2,0) 
node[circle,fill,inner sep=1pt,label=above:$u_2$](b){};
\draw (1.5,0.25) -- (1.5,-0.25);
\draw (2,0) node[circle,fill,inner sep=1pt](a){} -- (3,0) 
node[circle,fill,inner sep=1pt,label=above:$u_3$](b){};
\draw (2.5,0.25) -- (2.5,-0.25);
\draw (3,0) node[circle,fill,inner sep=1pt](a){} -- (4,0) 
node[circle,fill,inner sep=1pt,label=above:$u_4$](b){};
\draw (3.5,0.5) -- (3.5,-0.5);
\draw (3.5,0.2) node[draw=none,rectangle,label=above:$\Gamma$](a){};
\draw (3.5,0.4) node[draw=none,rectangle,label=below:${u|_{x=\Gamma}=\gamma}$](a){};
\draw (4,0) node[circle,fill,inner sep=1pt](a){} -- (5,0) 
node[circle,fill,inner sep=1pt,label=above:$u_5$](b){};
\draw (4.5,0.25) -- (4.5,-0.25);
\draw (5,0) node[circle,fill,inner sep=1pt](a){} -- (6,0) 
node[circle,fill,inner sep=1pt,label=above:$u_6$](b){};
\draw (5.5,0.25) -- (5.5,-0.25);
\draw (6,0) node[circle,fill,inner sep=1pt](a){} -- (7,0) 
node[circle,inner sep=2pt,label=above:$u_7$](b){};
\draw (6.5,0.5) -- (6.5,-0.5);
\draw (6.5,0.2) node[draw=none,rectangle,label=above:$1$](a){};
\draw (6.5,0) node[draw=none,rectangle,label=below:$u|_{x=1}$](a){};

   \end{tikzpicture}
     \caption{1-D line before domain decomposition. \label{fig:4.1}}
  \end{figure}

We can define $\gamma$ to be the average of its neighbor nodes as shown in equation \ref{eq:4.1}.

\begin{equation}\label{eq:4.1}
	\gamma = \frac{u_{j} + u_{j+1}}{2}
\end{equation}



In figure \ref{fig:4.2} we have decomposed the domain into two sub-domains, $\Omega_0$ and $\Omega_1$. Unlike the node-centered case, the two sub-domains do not share any nodes or receive any values for ghost nodes.

\begin {figure}[htbp]
  \centering
    \begin{tikzpicture}[>=triangle 60]
\tikzset{every node}=[draw,shape=circle]

\draw (0.5,-0.5)  -- (0.5,0.5)
node[draw=none,rectangle,label=above:$u^0|_{x=0}$](a){};
\draw (0.5,-0.2) node[draw=none,rectangle,label=below:$0$](a){};
\draw (0,0) node[circle,inner sep=2pt,label=above:$u_0^0$,label=left:\LARGE$\boldsymbol\Omega_0$](a){} -- (1,0) 
node[circle,fill,inner sep=1pt,label=above:$u_1^0$](b){};
\draw (1,0) node[circle,fill,inner sep=1pt](a){} -- (2,0) 
node[circle,fill,inner sep=1pt,label=above:$u_2^0$](b){};
\draw (1.5,0.25) -- (1.5,-0.25);
\draw (2,0) node[circle,fill,inner sep=1pt](a){} -- (3,0) 
node[circle,fill,inner sep=1pt,label=above:$u_3^0$](b){};
\draw (2.5,0.25) -- (2.5,-0.25);
\draw (3,0) node[circle,fill,inner sep=1pt](a){} -- (3.5,0);
\draw (3.5,-0.5) -- (3.5,0.5);
\draw (3.5,0.1) node[draw=none,rectangle,label=above:${u^0|_{x=\Gamma}=\gamma}$](a){};
\draw (3.5,-0.2) node[draw=none,rectangle,label=below:$\Gamma$](a){};

\draw[yshift=-1.5cm] (3.5,0) -- (4,0) 
node[circle,fill,inner sep=1pt,label=below:$u_{0}^1$](b){};
\draw[yshift=-1.5cm] (3.5,0.5) -- (3.5,-0.5);
\draw[yshift=-1.5cm] (3.5,-0.1) node[draw=none,rectangle,label=below:${u^1|_{x=\Gamma} = \gamma}$](a){};
\draw[yshift=-1.5cm] (4,0) node[circle,fill,inner sep=1pt](a){} -- (5,0) 
node[circle,fill,inner sep=1pt,label=below:$u_{1}^1$](b){};
\draw[yshift=-1.5cm] (4.5,0.25) -- (4.5,-0.25);
\draw[yshift=-1.5cm] (5,0) node[circle,fill,inner sep=1pt](a){} -- (6,0) 
node[circle,fill,inner sep=1pt,label=below:$u_{2}^1$](b){};
\draw[yshift=-1.5cm] (5.5,0.25) -- (5.5,-0.25);
\draw[yshift=-1.5cm] (6,0) node[circle,fill,inner sep=1pt](a){} -- (7,0)
node[circle,inner sep=2pt,label=below:$u_{4}^1$,label=right:\LARGE$\boldsymbol\Omega_1$](b){};
\draw[yshift=-1.5cm] (6.5,0.2) node[draw=none,rectangle,label=above:$1$](a){};
\draw[yshift=-1.5cm] (6.5,0.5) -- (6.5,-0.5)
node[draw=none,rectangle,label=below:$u^1|_{x=1}$](a){};


   \end{tikzpicture}
     \caption{1-D line after domain decomposition. \label{fig:4.2}}
  \end{figure}


This can be done by solving the Schur Complement system for the boundary values.