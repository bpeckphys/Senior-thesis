In high performance computing, one often deals with very large matrices that would be exorbitantly expensive to compute using direct solvers. One method of handling these matrices is to decompose them into smaller matrices that can then be approximated iteratively and in parallel. This method is known as domain decomposition. There are two ways of decomposing a domain into smaller sub-domains, overlapping and non-overlapping as-well-as two methods of evaluating these, node-centered and cell-centered. For this paper we will assume Dirichlet-Dirichlet boundary conditions.

In the first method, we will first use the node centered method and split the domain into smaller sub-domains that overlap each other such that the values at the boundaries between the sub-domains are equivalent. This method leaves the sub-domains coupled at the boundary and each of the sub-domains must communicate values to its adjacent sub-domains. There is some parallelism that is achieved using this method in a staggered fashion; however, the more sub-domains we decompose to, the more wait time there is for the sub-domains furthest from the first sub-domain to be approximated due to each sub-domain having to wait to start being approximated until receiving boundary values from the previous sub-domain. We can parallelize this method further by not communicating values between the sub-domains, but rather making an initial guess at the boundary value and then iterating until a Neumann boundary condition on the interior boundaries has been satisfied to within a given tolerance. This allows all of the sub-domains to be approximated at the same time instead of in a staggered fashion.

The second method decouples the sub-domains entirely allowing us to achieve full parallelism in solving the sub-domains. This is done by dividing the domain into sub-domains such that they share the boundary or boundaries without overlapping. If we solve for the values on the boundaries, then we can turn the single Dirichlet-Dirichlet boundary problem into an arbitrary number of smaller Dirichlet-Dirichlet problems. These boundary values can be found by solving the Schur-complement system.