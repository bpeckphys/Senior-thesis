\begin{figure}[h]
\centering
\includegraphics[scale=0.5]{graph_a}
\caption{An example graph}
\label{fig:x cubed graph}
\end{figure}
\ref{fig:x cubed graph}


\begin{figure}
    \centering
    \begin{subfigure}[b]{0.3\textwidth}
        \centering
        \includegraphics[width=\textwidth]{graph1}
        \caption{$y=x$}
        \label{fig:y equals x}
    \end{subfigure}
    \hfill
    \begin{subfigure}[b]{0.3\textwidth}
        \centering
        \includegraphics[width=\textwidth]{graph2}
        \caption{$y=3sinx$}
        \label{fig:three sin x}
    \end{subfigure}
    \hfill
    \begin{subfigure}[b]{0.3\textwidth}
        \centering
        \includegraphics[width=\textwidth]{graph3}
        \caption{$y=5/x$}
        \label{fig:five over x}
    \end{subfigure}
    \caption{Three simple graphs}
    \label{fig:three graphs}
\end{figure}


\begin{table}[h]
\centering
\begin{tabular}{l | l | l}
A & B & C \\
\hline
1 & 2 & 3 \\
4 & 5 & 6
\end{tabular}
\caption{very basic table}
\label{tab:abc}
\end{table}


\begin{table}[h]
    \begin{subtable}[h]{0.45\textwidth}
        \centering
        \begin{tabular}{l | l | l}
        Day & Max Temp & Min Temp \\
        \hline \hline
        Mon & 20 & 13\\
        Tue & 22 & 14\\
        Wed & 23 & 12\\
        Thurs & 25 & 13\\
        Fri & 18 & 7\\
        Sat & 15 & 13\\
        Sun & 20 & 13
        \end{tabular}
        \caption{First Week}
        \label{tab:week1}
    \end{subtable}
    \hfill
    \begin{subtable}[h]{0.45\textwidth}
        \centering
        \begin{tabular}{l | l | l}
        Day & Max Temp & Min Temp \\
        \hline \hline
        Mon & 17 & 11\\
        Tue & 16 & 10\\
        Wed & 14 & 8\\
        Thurs & 12 & 5\\
        Fri & 15 & 7\\
        Sat & 16 & 12\\
        Sun & 15 & 9
        \end{tabular}
        \caption{Second Week}
        \label{tab:week2}
    \end{subtable}
    \caption{Max and min temps recorded in the first two weeks of July}
    \label{tab:temps}
\end{table}


\cite{latexcompanion}


TEXT\parencite[see][p10]{latexcompanion}
TEXT\parencite[compare][]{knuthwebsite}
TEXT\parencite[e.g.][page 300]{einstein}


\begin{equation*}
\begin{matrix}
1 & 2 \\
3 & 4
\end{matrix} \qquad
\begin{bmatrix}
p_{11} & p_{12} & \ldots
& p_{1n} \\
p_{21} & p_{22} & \ldots
& p_{2n} \\
\vdots & \vdots & \ddots
& \vdots \\
p_{m1} & p_{m2} & \ldots
& p_{mn}
\end{bmatrix}
\end{equation*}

















